% ----------------------------------------------------------
\chapter{Ferramentas e Tecnologias Utilizadas}\label{cap:tecnologias}
% ----------------------------------------------------------

Neste capítulo são apresentadas as ferramentas e tecnologias utilizadas no desenvolvimento do sistema.

\section{Ferramentas e Tecnologias Utilizadas}\label{sec:ferramentas_tecnologias}

A seção de ferramentas e tecnologias utilizadas no desenvolvimento deste projeto apresenta um panorama detalhado dos recursos técnicos empregados para a construção da aplicação de servidor do sistema de gerenciamento de reservas e controle automatizado de dispositivos IoT para quadras esportivas. 
% O projeto foi desenvolvido utilizando uma stack moderna e robusta, com o \textit{backend} implementado em Nest.js, um framework eficiente e escalável para Node.js, e o \textit{frontend} será desenvolvido em React, a fim de proporcionar uma experiência de usuário dinâmica e responsiva. Além disso, o banco de dados relacional PostgreSQL foi utilizado para garantir a integridade e a consistência dos dados, enquanto a integração com a plataforma eWeLink permitiu o controle automatizado de dispositivos IoT, agregando um nível adicional de automação e inovação ao sistema. Este conjunto de ferramentas foi escolhido com o objetivo de garantir um desenvolvimento ágil, eficiente e que atendesse às necessidades específicas do projeto.

\subsection{Git}

O Git é um sistema de controle de versão distribuído criado por Linus Torvalds em 2005, inicialmente para o desenvolvimento do kernel do Linux. Ele permite rastrear alterações no código-fonte, possibilitando que múltiplos desenvolvedores colaborem em um mesmo projeto sem conflitos. Suas principais funcionalidades incluem a criação de ramificações (\textit{branches}) para desenvolvimento isolado, a segurança proporcionada pela cópia distribuída dos repositórios e a eficiência na integração de alterações. Atualmente, o Git é a ferramenta de controle de versão mais popular do mundo, sendo amplamente adotada em projetos empresariais e de código aberto.  

O GitHub é uma plataforma de hospedagem de código que utiliza o Git para controle de versão, facilitando a colaboração entre desenvolvedores ao redor do mundo. Com recursos como \textit{pull requests} e \textit{issues}, permite a contribuição em projetos privados e públicos, simplificando a gestão de código e a comunicação entre equipes. A plataforma abriga milhões de projetos e conta com uma vasta comunidade de usuários ativos, tornando-se essencial no ecossistema do desenvolvimento de software.  

A Fischertec utiliza o Git e o GitHub como padrão para o versionamento de código e a colaboração entre equipes. A escolha do Git deve-se à sua robustez, confiabilidade e familiaridade da equipe com a ferramenta. Já o GitHub foi adotado por ser amplamente reconhecido e oferecer funcionalidades essenciais para a organização e divulgação de projetos, mantendo-se alinhado às práticas utilizadas na empresa.  

O uso dessas ferramentas garante um fluxo de trabalho eficiente, seguro e colaborativo, permitindo que o desenvolvimento de software ocorra de forma estruturada, organizada e integrada, desde pequenas aplicações até grandes sistemas empresariais.

\subsection{PostgreSQL}

O \acrlong{SQL} foi desenvolvido nos anos 1970 pela IBM como uma linguagem padrão para gerenciar e manipular bancos de dados relacionais. PostgreSQL, um sistema de gerenciamento de banco de dados relacional, surgiu na Universidade da Califórnia em Berkeley, no final dos anos 1980, como parte do projeto POSTGRES (Post Ingres). Em 1996, o sistema foi renomeado para PostgreSQL, refletindo sua compatibilidade com SQL. Desde então, PostgreSQL evoluiu significativamente, tornando-se uma das bases de dados relacionais mais robustas e avançadas disponíveis.

PostgreSQL oferece uma ampla gama de funções, incluindo suporte a transações ACID (Atomicidade, Consistência, Isolamento, Durabilidade), integridade referencial, e extensões para manipulação de dados complexos como JSON e XML. Além disso, suporta procedimentos armazenados, gatilhos e uma variedade de tipos de índices para otimizar consultas. Atualmente, PostgreSQL é amplamente utilizado por grandes empresas, startups e projetos de código aberto devido à sua confiabilidade, flexibilidade e conformidade com os padrões SQL.

Existem outras soluções de banco de dados SQL como MySQL, que é conhecido por sua facilidade de uso e performance em aplicações web, e soluções NoSQL como MongoDB, que é ideal para armazenar grandes volumes de dados não estruturados e aplicações que exigem alta escalabilidade e flexibilidade. Optou-se por usar o PostgreSQL no desenvolvimento do projeto devido à sua capacidade de lidar com transações complexas, suporte a dados estruturados e não estruturados, e forte conformidade com os padrões SQL, o que garante integridade e consistência dos dados.

O PostgreSQL se destaca como uma solução de banco de dados robusta e versátil, adequada para uma ampla gama de aplicações. Sua evolução ao longo dos anos e a capacidade de suportar funcionalidades avançadas o tornam uma escolha excelente para projetos que exigem confiabilidade e flexibilidade. A decisão de utilizá-lo no projeto foi fundamentada em sua capacidade de atender às necessidades específicas de gerenciamento de dados de maneira eficiente e segura.

\subsection{Typescript}

O JavaScript, criado em 1995 por Brendan Eich da Netscape, rapidamente se tornou a linguagem padrão para desenvolvimento web, permitindo a criação de páginas dinâmicas e interativas. No entanto, à medida que aplicações web se tornavam mais complexas, as limitações de JavaScript em termos de tipagem estática e suporte para grandes projetos se tornaram evidentes. Para reduzir essas limitações, a Microsoft lançou o TypeScript em 2012, uma linguagem de programação de código aberto que é um superconjunto (superset) estrito de JavaScript. TypeScript adiciona tipagem estática e outros recursos avançados ao JavaScript, ajudando os desenvolvedores a escrever código mais seguro e escalável.

As principais funções do TypeScript incluem a adição de tipos estáticos, interfaces, classes e módulos, que não existem nativamente no JavaScript. Essas funcionalidades permitem a detecção de erros durante o desenvolvimento, antes do tempo de execução, melhorando a qualidade do código. Além disso, TypeScript se transpila (é convertido) para JavaScript, garantindo compatibilidade total com os navegadores e plataformas que suportam JavaScript. Atualmente, TypeScript é amplamente adotado em projetos de grande escala devido à sua capacidade de melhorar a produtividade e a manutenção do código. Empresas como Google, Microsoft e Airbnb utilizam TypeScript em seus projetos.

No desenvolvimento do projeto \textit{backend}, optou-se por TypeScript em vez de JavaScript por várias razões. TypeScript proporciona uma melhor experiência de desenvolvimento, oferecendo autocompletar, navegação de código e verificação de tipos em tempo de compilação, o que ajuda a evitar muitos erros comuns em JavaScript. Isso é especialmente importante em projetos \textit{backend}, onde a robustez e a previsibilidade do código são cruciais. Além disso, TypeScript facilita a colaboração entre desenvolvedores, permitindo que eles entendam e mantenham o código com mais facilidade.

O TypeScript se destaca como uma ferramenta poderosa que complementa e aprimora JavaScript, tornando o desenvolvimento de software mais eficiente e menos propenso a erros. Sua adoção no desenvolvimento da aplicação \textit{backend} de gerenciamento de reservas e controle automatizado de dispositivos IoT permitiu criar um código mais robusto e sustentável.

\subsection{Node.js}

O Node.js é uma plataforma de desenvolvimento open-source baseada no motor JavaScript V8 do Google Chrome, criada em 2009 por Ryan Dahl. Seu objetivo inicial era possibilitar a execução de JavaScript no lado do servidor, permitindo o desenvolvimento de aplicações web com maior eficiência e escalabilidade. Desde seu lançamento, o Node.js rapidamente ganhou popularidade devido à sua arquitetura orientada a eventos e sua capacidade de lidar com operações de entrada e saída de forma não bloqueante, o que é particularmente útil para aplicações em tempo real.

Atualmente, o Node.js é amplamente utilizado para construir desde APIs e microservices até aplicações completas de grande escala. Sua capacidade de executar JavaScript tanto no cliente quanto no servidor facilita o desenvolvimento fullstack, enquanto a vasta biblioteca de pacotes disponíveis através do npm (Node Package Manager) oferece soluções para praticamente qualquer necessidade de desenvolvimento. O Node.js tem sido uma escolha popular em diversas indústrias devido à sua performance, escalabilidade e ao suporte contínuo de uma grande comunidade de desenvolvedores.

Uma das principais vantagens do Node.js em comparação com outras soluções é sua natureza assíncrona e orientada a eventos, que permite o gerenciamento eficiente de múltiplas conexões simultâneas sem sobrecarregar os recursos do servidor. Além disso, a utilização de JavaScript, uma linguagem amplamente conhecida e utilizada, reduz a curva de aprendizado para novos desenvolvedores e facilita a integração entre equipes de \textit{frontend} e \textit{backend}. A modularidade do Node.js e sua grande comunidade de apoio também contribuem para o desenvolvimento mais rápido e eficiente de aplicações.

Portanto o Node.js se destaca como uma solução versátil e eficiente para o desenvolvimento de aplicações modernas. Sua combinação de performance, escalabilidade e uma vasta gama de ferramentas e bibliotecas tornam-no uma escolha robusta para desenvolvedores que buscam criar aplicações rápidas e escaláveis, atendendo às demandas de um mercado cada vez mais dinâmico e competitivo.

\subsection{Nest.js}

O Nest.js é um framework de desenvolvimento \textit{backend} criado em 2017 por Kamil Myśliwiec. Inspirado nos princípios de programação modular e fortemente influenciado pelo Angular, o Nest.js foi projetado para oferecer uma estrutura robusta e escalável para a construção de aplicações do lado do servidor. Desde o seu lançamento, o framework tem crescido em popularidade, especialmente entre desenvolvedores que buscam uma abordagem moderna para o desenvolvimento de aplicações \textit{backend} em Node.js.

O Nest.js fornece uma arquitetura modular que facilita a criação e a manutenção de aplicações escaláveis e bem estruturadas. Ele suporta diversos paradigmas de programação, incluindo orientação a objetos, programação funcional e reativa. Com suporte nativo para TypeScript, o Nest.js oferece tipagem estática, o que melhora a confiabilidade e a legibilidade do código. Atualmente, é amplamente utilizado para construir APIs RESTful, microservices e aplicativos monolíticos, sendo uma escolha frequente em projetos que exigem alta escalabilidade e flexibilidade.

Entre as principais vantagens do Nest.js estão sua modularidade, que permite a criação de aplicações altamente organizadas e de fácil manutenção, e seu suporte integral a TypeScript, que traz maior segurança e produtividade no desenvolvimento. Em comparação com outros frameworks, como Express.js, o Nest.js oferece uma estrutura mais robusta e orientada a boas práticas, facilitando o desenvolvimento de aplicações complexas. No contexto do \acrlong{PFC}, o Nest.js foi escolhido por sua capacidade de suportar a criação de uma aplicação \textit{multi-tenant} complexa, com necessidades específicas de escalabilidade, organização e integração com outras tecnologias.

O Nest.js se destaca como uma ferramenta poderosa para o desenvolvimento de aplicações \textit{backend} modernas, combinando a performance do Node.js com uma arquitetura flexível e escalável. Sua adoção neste \acrlong{PFC} reflete a busca por soluções eficientes e de alta qualidade, alinhadas com as demandas atuais do mercado de tecnologia.

\subsection{Docker}

O Docker foi lançado em 2013 por Solomon Hykes, inicialmente como um projeto interno da empresa dotCloud. A plataforma foi criada com o objetivo de simplificar o processo de desenvolvimento, distribuição e execução de aplicações por meio da virtualização de contêineres. Desde seu lançamento, o Docker tem revolucionado a maneira como aplicações são desenvolvidas e implantadas, tornando-se uma das ferramentas mais populares no ecossistema de DevOps.

Atualmente, o Docker é amplamente utilizado para criar, distribuir e executar aplicações em contêineres, que são ambientes leves e portáteis que contêm tudo o que uma aplicação precisa para ser executada. Isso inclui código, bibliotecas e dependências, garantindo que a aplicação funcione de maneira consistente em qualquer ambiente. O Docker é uma escolha comum em projetos que exigem portabilidade, escalabilidade e uma integração contínua eficiente, sendo adotado por empresas de todos os tamanhos para modernizar seus fluxos de trabalho de desenvolvimento e implantação.

Entre as principais vantagens do Docker estão a portabilidade de aplicações, a eficiência no uso de recursos e a facilidade de integração com pipelines de CI/CD (integração contínua e entrega contínua). Comparado com máquinas virtuais tradicionais, os contêineres do Docker são mais leves e rápidos, o que resulta em tempos de inicialização menores e melhor utilização de recursos. No contexto deste projeto, o Docker foi escolhido para garantir que o ambiente de desenvolvimento e produção seja consistente, facilitando a implantação em servidores AWS e reduzindo problemas de compatibilidade.

O Docker se destaca como uma solução essencial para o desenvolvimento e a implantação de aplicações modernas, oferecendo uma maneira eficiente de gerenciar ambientes de software. Sua escolha neste projeto reflete a busca por uma solução que ofereça portabilidade, eficiência e flexibilidade, alinhando-se às melhores práticas do mercado e garantindo um processo de desenvolvimento mais ágil e confiável.

\subsection{Postman}

O Postman foi lançado em 2012 por Abhinav Asthana como uma ferramenta auxiliar para o desenvolvimento de APIs. Inicialmente criado como uma extensão para o Google Chrome, o Postman rapidamente evoluiu para uma aplicação completa e independente, tornando-se uma das ferramentas mais populares para o teste e desenvolvimento de APIs. Ao longo dos anos, a plataforma expandiu suas funcionalidades para atender às crescentes demandas de desenvolvedores e equipes de API.

O Postman é amplamente utilizado para testar, documentar e monitorar APIs, oferecendo uma interface amigável que facilita a realização de requisições HTTP, o gerenciamento de coleções de APIs e a automatização de testes. Além disso, ele permite a colaboração em equipe, possibilitando o compartilhamento de coleções e ambientes de teste. Atualmente, o Postman é uma ferramenta indispensável no fluxo de trabalho de desenvolvedores e engenheiros de qualidade, sendo utilizado por milhões de usuários ao redor do mundo.

As vantagens do Postman incluem sua interface intuitiva, que reduz a complexidade do teste de APIs, e suas capacidades avançadas de automação e documentação. Comparado a outras ferramentas, como cURL ou alternativas mais básicas, o Postman oferece uma experiência mais integrada e acessível, especialmente para equipes que precisam colaborar em projetos complexos. No desenvolvimento deste projeto, o Postman foi escolhido por sua facilidade de uso e suas capacidades de automação de testes, o que contribui para um processo de desenvolvimento mais eficiente e menos propenso a erros.

Portanto o Postman se consolida como uma ferramenta essencial para o desenvolvimento e manutenção de APIs, oferecendo funcionalidades abrangentes que facilitam o trabalho de desenvolvedores e equipes de qualidade. Sua utilização neste projeto destaca a importância de ferramentas que otimizam o fluxo de trabalho, garantindo maior eficiência e qualidade no desenvolvimento de sistemas modernos.