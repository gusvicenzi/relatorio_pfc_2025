% ----------------------------------------------------------
\chapter{Fundamentação Teórica}\label{cap:desenvolvimento}
% ----------------------------------------------------------

\textbf{Instruções da Coordenação do PFC:}

Deve-se colocar um parágrafo introdutório no início de \textbf{cada} capítulo, descrevendo os assuntos que serão abordados e a relação com o restante do trabalho. Por exemplo: \emph{A Seção 2.1 apresenta \ldots. Os resultados obtidos são analisados na Seção 2.2.} Pode-se fazer o mesmo no início de seções maiores, explicando para o leitor, em uma ou duas sentenças, o que está por vir no texto e o porquê. Outra boa prática é, ao final de cada capítulo, fazer uma ligação com o capítulo seguinte por meio de parágrafo curto.

Neste capítulo, deve-se apresentar as principais teorias, conceitos, técnicas, modelos, etc que são essenciais para o entendimento do problema tratado e da solução proposta.

Figuras, tabelas, quadros e equações devem ser introduzidos e explicados no texto: não se pode simplesmente ``jogá-los'' no texto, sem referência nem explicação. Por exemplo, deve-se escrever algo como: \emph{O circuito projetado é mostrado na Figura~10. O resistor $R_1$ faz o papel de um limitador de corrente, enquanto o capacitor $C_2$ juntamente com o resistor $R_5$ formam um filtro passa-baixa. Este circuito tem a vantagem de \ldots}

Com relação às equações, não se faz referência a uma equação que ainda não foi apresentada. Por exemplo, não se escreve: \emph{A relação entre a tensão e a corrente de um resistor é dada pela \autoref{eq:leiDeOhm}:}
\begin{equation}\label{eq:leiDeOhm}
    V = R I \, \text{.}
\end{equation}

\noindent O correto é algo como: \emph{A relação entre a tensão e a corrente de um resistor é dada por (Lei de Ohm)}
\begin{equation}\label{eq:leiDeOhm2}
    V = R I \, \text{,}
   \end{equation}
\emph{na qual $V$ é a tensão sobre o resistor, $R$ a resistência e $I$ a corrente elétrica. Da \autoref{eq:leiDeOhm2}, obtemos que}
\begin{equation}\label{eq:leiDeOhm3}
    I = \dfrac{V}{R} \, \text{.}
\end{equation}

\emph{Por outro lado, \ldots}

É importante observar que as equações fazem parte do texto e, assim, deve-se inserir uma vírgula ou ponto ao seu final. Se o parágrafo segue, pode-se eliminar o recuo na próxima linha com o comando \verb!\noindent!. Além disto, se a frase segue, inicia-se a linha com letra minúscula. Veja os exemplos das Equações~(\ref{eq:leiDeOhm2}) e (\ref{eq:leiDeOhm3}).

A seguir encontra-se uma equação na linha de texto: $\hat{y}(t+k\mid t)= \sum^\infty_{i=1} g_i \Delta u(t+k-i\mid t)$. E também, mais à frente, um exemplo de referência cruzada da \autoref{fig:Fig_1} e da \autoref{eq:Eq_1}.

\pagebreak

\textbf{Instruções do padrão genérico de TCCs da BU:}

Deve-se inserir texto entre as seções.

% ----------------------------------------------------------
\section{Exposição do tema ou matéria}
% ----------------------------------------------------------

É a parte principal e mais extensa do trabalho. Deve apresentar a fundamentação teórica, a metodologia, os resultados e a discussão. Divide-se em seções e subseções conforme a NBR 6024 \cite{NBR6024:2012}.

Quanto à sua estrutura e projeto gráfico, segue as recomendações da \gls{ABNT} para preparação de trabalhos acadêmicos, a NBR 14724, de 2011 \cite{NBR14724:2011}.

\begin{figure}[htb]
	\caption{\label{fig:Fig_1}Elementos do trabalho acadêmico.}
	\begin{center}
		\includegraphics{images/imagem.pdf}
	\end{center}
	\fonte{Universidade Federal do Paraná (1996).}
\end{figure}

% ----------------------------------------------------------
\subsection{Formatação do texto}
% ----------------------------------------------------------

No que diz respeito à estrutura do trabalho, recomenda-se que:
\begin{alineas}
	\item o texto deve ser justificado, digitado em cor preta, podendo utilizar outras cores somente para as ilustrações;
	\item utilizar papel branco ou reciclado para impressão;
	\item os elementos pré-textuais devem iniciar no anverso da folha, com exceção da ficha catalográfica ou ficha de identificação da obra;
	\item os elementos textuais e pós-textuais devem ser digitados no anverso e verso das folhas, quando o trabalho for impresso. As seções primárias devem começar sempre em páginas ímpares, quando o trabalho for impresso. Deixar um espaço entre o título da seção/subseção e o texto e entre o texto e o título da subseção.
\end{alineas}

No \autoref{qua:Quadro_1} estão as especificações para a formatação do texto.

\begin{quadro}[htb]
	\centering
	\caption{\label{qua:Quadro_1}Formatação do texto.}	
	\begin{tabular}{|l|p{11cm}|}
		\hline
		\textbf{Formato do papel} & A4.\\ \hline
		\textbf{Impressão}        & A norma recomenda que caso seja necessário imprimir, deve-se utilizar a frente e o verso da página.\\ \hline
		\textbf{Margens}          & Superior: 3, Inferior: 2, Interna: 3 e Externa: 2. Usar margens espelhadas quando o  trabalho for impresso.\\ \hline
		\textbf{Paginação}        & As páginas dos elementos pré-textuais devem ser contadas, mas não numeradas. Para trabalhos digitados somente no anverso, a numeração das páginas deve constar no canto superior direito da página, a 2 cm da borda, figurando a partir da primeira folha da  parte textual. Para trabalhos digitados no anverso e no verso, a numeração deve constar no canto superior direito, no anverso, e no canto superior esquerdo no verso.\\ \hline
		\textbf{Espaçamento}      & O texto deve ser redigido com espaçamento entre linhas 1,5, excetuando-se as citações de mais de três linhas, notas de rodapé, referências, legendas das ilustrações e das tabelas, natureza (tipo do trabalho, objetivo, nome da instituição a que é submetido e área de concentração), que devem ser digitados em espaço simples, com fonte menor. As referências devem ser separadas entre si por um espaço simples em branco.\\ \hline
		\textbf{Paginação}        & A contagem inicia na folha de rosto, mas se insere o número da página na introdução até o final do trabalho.\\ \hline
		\textbf{Fontes sugeridas} & Arial ou Times New Roman.\\ \hline
		\textbf{Tamanho da fonte} & \textbf{Fonte tamanho 12 para o texto}, incluindo os títulos das seções e subseções. As citações com mais de três linhas, notas de rodapé, paginação, dados internacionais de catalogação, legendas e fontes das ilustrações e das tabelas devem ser de tamanho menor. Adotamos, neste \textit{template} \textbf{fonte tamanho 10}.\\ \hline
		\textbf{Nota de rodapé}   & Devem ser digitadas dentro da margem, ficando separadas por um espaço simples por entre as linhas e por filete de 5 cm a partir da margem esquerda. A partir da segunda linha, devem ser alinhadas embaixo da primeira letra da primeira palavra da primeira linha.\\ \hline
	\end{tabular}
	\fonte{\textcite{NBR14724:2011}.}
\end{quadro}

% ----------------------------------------------------------
\subsubsection{As ilustrações}
% ----------------------------------------------------------

Independentemente do tipo de ilustração (quadro, desenho, figura, fotografia, mapa, entre outros), a sua identificação aparece na parte superior, precedida da palavra designativa. 

\begin{citacao}
	Após a ilustração, na parte inferior, indicar a fonte consultada (elemento obrigatório, mesmo que seja produção do próprio autor), legenda, notas e outras informações necessárias à sua compreensão (se houver). A ilustração deve ser citada no texto e inserida o mais próximo possível do texto a que se refere. \cite[p. 11]{NBR14724:2011}.
\end{citacao}

% ----------------------------------------------------------
\subsubsection{Equações e fórmulas}
% ----------------------------------------------------------

As equações e fórmulas devem ser destacadas no texto para facilitar a leitura.  Para numerá-las, usar algarismos arábicos entre parênteses e alinhados à direita. Pode-se adotar uma entrelinha maior do que a usada no texto \cite{NBR14724:2011}.

Exemplos, \autoref{eq:Eq_1} e \autoref{eq:Eq_2}. Observe que o comando \verb|\gls{}| é usado para utilizar para criar um \emph{hyperlink} com a definição do símbolo na lista de símbolos (veja linha 153 de \emph{main.tex}.

\begin{equation}
\label{eq:Eq_1}
\gls{C} = 2 \gls{pi} \gls{r} \sqrt{\gamma} + 10 \, \text{.}
\end{equation}

\begin{equation}
\label{eq:Eq_2}
\gls{A} = \gls{pi} \gls{r}^2 \, \text{.}
\end{equation}

\noindent Aqui não há recuo porque o parágrafo não terminou, apenas foi iniciada uma nova frase após a equação. As equações fazem parte do texto, portanto estão sujeitas à pontuação (ponto final, vírgula etc.).

% ----------------------------------------------------------
\subsubsubsection{Exemplo tabela}
% ----------------------------------------------------------

De acordo com \textcite{ibge1993}, tabela é uma forma não discursiva de apresentar informações em que os números representam a informação central. Ver \autoref{tab:Tab_1}.

\begin{table}[htb]
	\ABNTEXfontereduzida
	\caption{\label{tab:Tab_1}Médias concentrações urbanas 2010-2011.}
	\begin{tabular}{@{}p{3.0cm}p{1.5cm}p{2cm}p{2.5cm}p{2.5cm}p{2.5cm}@{}}
		\toprule
		\textbf{Média concentração urbana} & \multicolumn{2}{l}{\textbf{População}} & \textbf{Produto Interno Bruto – PIB (bilhões R\$)} & \textbf{Número de empresas} & \textbf{Número de unidades locais} \\ \midrule
		\textbf{Nome}                      & \textbf{Total}   & \textbf{No Brasil}  &                                                   &                             & \\
		Ji-Paraná (RO)                     & 116 610          & 116 610             & 1,686                                             & 2 734                       & 3 082 \\
		Parintins (AM)                     & 102 033          & 102 033             & 0,675                                             & 634                         & 683 \\
		Boa Vista (RR)                     & 298 215          & 298 215             & 4,823                                             & 4 852                       & 5 187 \\
		Bragança (PA)                      & 113 227          & 113 227             & 0,452                                             & 654                         & 686 \\ \bottomrule
	\end{tabular}
	\fonte{\textcite{ibge2016}.}
\end{table}

\section{APIs e APIs Restful}
\section{IoT}
\section{Protocolos Wifi e Zigbee}
\section{Software as a Service (SaaS)}
\section{Conteineres}
\section{Arquitetura em Camadas}

A arquitetura em camadas é um modelo amplamente adotado no desenvolvimento de aplicações modernas, proporcionando uma estrutura clara e organizada que facilita a manutenção, escalabilidade e reutilização de código. Esta abordagem divide a aplicação em camadas distintas, cada uma com responsabilidades específicas e bem definidas. No desenvolvimento da aplicação, a arquitetura em camadas é utilizada para separar as preocupações, garantindo que cada componente funcione de forma independente e coesa. \cite{Fowler02} descreve a arquitetura em camadas como uma abordagem fundamental para a construção de aplicações corporativas, destacando sua popularidade e uso extensivo em aplicações empresariais. Cada uma das camadas é apresentada da \autoref{subsec:camada_apresentacao} à \autoref{subsec:camada_banco_de_dados}.

\subsection{Camada de Apresentação}\label{subsec:camada_apresentacao}

A camada de apresentação é responsável pela interface com o usuário final e pode ser desenvolvida utilizando diversas tecnologias e frameworks, como Next.js com a biblioteca React.js, Vue.js, Angular.js, HTML e CSS entre outros. O Next.js, por exemplo, com sua capacidade de renderização híbrida (SSR e SSG), oferece uma experiência de usuário rápida e otimizada para SEO, enquanto o React possibilita a criação de componentes reutilizáveis e uma interface dinâmica e responsiva. Nesta camada, as interações do usuário são capturadas e encaminhadas para a camada de aplicação geralmente através de chamadas HTTP a APIs RESTful, garantindo uma separação clara entre a interface e a lógica de negócios, porém outros protocolos podem ser usados para a comunicação.

\subsection{Camada de Aplicação}

A camada de aplicação atua como intermediária entre a interface do usuário e a lógica de negócios. Executando no servidor, esta camada gerencia a lógica de controle e orquestra as operações entre a camada de apresentação e a camada de negócios. Diversas tecnologias podem ser usadas para o desenvolvimento da aplicação tais como Node.js, Nest.js, Java, Spring, C\#, .NET, PHP, Python entre outros. Nesta camada, são definidos controladores que lidam com as requisições recebidas, e as enchaminham para a camada de negócios adequada para o processamento dos dados.

\subsection{Camada de Negócios}

A camada de negócios encapsula a lógica principal do sistema, garantindo que as regras de negócios sejam aplicadas de forma consistente. Esta camada é responsável pela implementação das regras de autenticação e autorização de usuários, além de toda a lógica necessária para o funcionamento adequado da aplicação. Ao concentrar as regras de negócio nesta camada, a aplicação assegura que todas as operações críticas são tratadas de forma centralizada e independente das outras camadas, simplificando o desenvolvimento, manutenção e melhoria do código da aplicação.

\subsection{Camada de Persistência}

A camada de persistência é responsável por gerenciar a interação com o banco de dados de forma abstrata ou direta. Pode-se utilizar tecnologias como TypeORM e Prisma nesta camada para abstrair a comunicação ou usar diretamente as \textit{queries} para interação com o banco de dados, permitindo realizar operações de criação, leitura, atualização e exclusão de dados de forma simplificada e eficiente.

\subsection{Camada de Banco de Dados}\label{subsec:camada_banco_de_dados}

A camada de banco de dados é o próprio sistema de armazenameto escolhido para o projeto, podendo ser um banco de dados estruturado (como PostgreSQL, MySQL, Oracle entre outro) ou não (como MongoDB, Cassandran InfluxDB, etc ). Esta camada é responsável por armazenar e recuperar os dados persistidos pela camada de persistência. A escolha adequada de um sistema de banco de dados garante a escalabilidade, consistência e disponibilidade dos dados, além de permitir o uso de recursos avançados para otimizar o desempenho.

\subsection{Benefícios da Arquitetura em Camadas}

A arquitetura em camadas traz vários benefícios para o desenvolvimento desta aplicação. A separação de preocupações facilita a manutenção do código, permitindo que alterações em uma camada específica não afetem diretamente as demais. Isso também melhora a escalabilidade da aplicação, pois novas funcionalidades podem ser adicionadas ou modificadas de forma isolada. Além disso, a modularização contribui para uma melhor reutilização de componentes, tornando o desenvolvimento mais eficiente e ágil. \cite{Martin17} discute como a arquitetura em camadas permite uma divisão clara de responsabilidades, promovendo independência no desenvolvimento e manutenção.

Portanto a adoção de uma arquitetura em camadas proporciona uma estrutura organizada e clara, que facilita o desenvolvimento, manutenção e evolução da aplicação. Cada camada desempenha um papel essencial, garantindo que a aplicação seja robusta, escalável e fácil de manter. A utilização desta abordagem, aliada às tecnologias escolhidas, assegura a entrega de um sistema eficiente, confiável e alinhado às melhores práticas do desenvolvimento de software moderno.

\section{Arquitetura Modular}
