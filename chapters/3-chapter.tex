% ----------------------------------------------------------
\chapter{Descrição do Problema e Requisitos Técnicos}\label{cap:descricao_problema_e_requisitos}
% ----------------------------------------------------------

\textbf{Instruções da Coordenação do PFC:}

Neste capítulo, deve-se apresentar (de forma mais detalhada e aprofundada tecnicamente que na Introdução):
\begin{itemize}
	\item O contexto e a motivação do PFC;
	\item Descrição da empresa/instituto de pesquisa (histórico, clientes, produtos, serviços, projetos, etc) em que o PFC foi realizado e do projeto global da empresa em que o PFC está inserido (se for o caso);
     \item Descrição do problema tratado no PFC;
     \item Requisitos técnicos (funcionais e não-funcionais).
\end{itemize}

Procure utilizar equações, tabelas, diagramas e fluxogramas para ilustrar e explicar melhor as ideias.


\section{Contextualização}

Agora começarei o capítulo de descrição do problema e requisitos técnicos. Neste capítulo, deve-se apresentar (de forma mais detalhada e aprofundada tecnicamente que na Introdução:
O contexto e a motivação do PFC;
Descrição da empresa/instituto de pesquisa (histórico, clientes, produtos, serviços, projetos, etc) em que o PFC foi realizado e do projeto global da empresa em que o PFC está inserido (se for o caso);
Descrição do problema tratado no PFC;
Requisitos técnicos (funcionais e não-funcionais)

A primeira seção será de contextualização
\section{Descrição do problema}
\section{solução proposta}
\section{Requisitos Técnicos a serem atendidos}
