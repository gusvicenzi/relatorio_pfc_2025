% ----------------------------------------------------------
\chapter{Introdução}
% ----------------------------------------------------------

\textbf{Instruções do padrão genérico de TCCs da BU:} 

As orientações aqui apresentadas são baseadas em um conjunto de normas elaboradas pela \gls{ABNT}. Além das normas técnicas, a Biblioteca também elaborou uma série de tutoriais, guias, \textit{templates} os quais estão disponíveis em seu site, no endereço \url{http://portal.bu.ufsc.br/normalizacao/}.

Paralelamente ao uso deste \textit{template} recomenda-se que seja utilizado o \textbf{Tutorial de Trabalhos Acadêmicos} (disponível neste link \url{https://repositorio.ufsc.br/handle/123456789/180829}).

Este \textit{template} está configurado apenas para a impressão utilizando o anverso das folhas, caso você queira imprimir usando a frente e o verso, acrescente a opção \textit{openright} e mude de \textit{oneside} para \textit{twoside} nas configurações da classe \textit{abntex2} no início do arquivo principal \textit{main.tex} \cite{abntex2classe}.

Conforme a \href{https://repositorio.ufsc.br/bitstream/handle/123456789/197121/RN46.2019.pdf?sequence=1&isAllowed=y}{Resolução NORMATIVA nº 46/2019/CPG} as dissertações e teses não serão mais entregues em formato impresso na Biblioteca Universitária. Consulte o Repositório Institucional da UFSC ou sua Secretaria de Pós Graduação sobre os procedimentos para a entrega. 

\nocite{NBR6023:2002}
\nocite{NBR6027:2012}
\nocite{NBR6028:2003}
\nocite{NBR10520:2002}

\textbf{Instruções da Coordenação do PFC:} 

A Introdução deve apresentar um panorama geral do PFC de modo resumido, sem entrar em detalhes técnicos, para que o leitor já entenda aqui na Introdução o começo, meio e fim do PFC. Assim, é muito importante deixar bem claro na Introdução os seguintes pontos (a Introdução pode ser vista como um resumo geral do PFC, e a Seção Resumo como um resumo desse resumo): 

\begin{itemize}
     \item O contexto e a motivação do PFC, com uma breve descrição da empresa/instituto de pesquisa (histórico, clientes, produtos, serviços, projetos, etc) em que o PFC foi realizado e do projeto global da empresa em que o PFC está inserido (se for o caso)
     \item Breve descrição do problema tratado no PFC;
    \item A importância de tal problema para a empresa/clientes da empresa/instituto de pesquisa;
    \item Objetivos: aqui são descritos os objetivos, que podem ser estratificados em objetivo geral e objetivos específicos. Outra opção é colocar os objetivos específicos na forma de passos de uma metodologia de trabalho, ou seja, os procedimentos e ferramentas adotadas em cada fase do projeto (um plano de trabalho).
    \item Breve descrição da solução proposta;
    \item Breve descrição da implementação/desenvolvimento realizado;
    \item Breve descrição da metodologia e das principais técnicas e ferramentas utilizadas (do ponto de vista técnico). \textbf{Importante}: como \emph{Scrum} é uma metodologia geral de execução de projetos, ele não se enquadra aqui, pois não se trata de uma metodologia técnica específica para o desenvolvimento do PFC;
    \item Breve descrição dos principais resultados obtidos e da importância/impactos de tais resultados para a empresa/clientes da empresa/instituto de pesquisa;
    \item Deixar bem claro o que foi de fato foi realizado pelo(a) autor(a), diferenciando do que foi aproveitado de trabalhos anteriores/outras times da empresa. \textbf{Importante}: esta preocupação em diferenciar o trabalho realizado pelo(a) autor(a) daquele de possíveis colegas de um time deve permear todo o documento.
\end{itemize}

Ressaltamos a seguir alguns aspectos cruciais para a escrita do PFC:
\begin{itemize}
	\item Apesar de o tamanho da monografia não ter uma correlação direta com a nota, bons trabalhos costumam ter de 60 a 70 páginas efetivamente escritas (desde a Introdução até a Conclusão, excluindo-se Capa, Resumo, Sumário, Referências Bibliográficas, etc).  Por outro lado, acima de 100 páginas a monografia pode se tornar ``massante'', discorrendo além do necessário para o entendimento do trabalho e, consequentemente, perdendo o foco do leitor.
	\item A linguagem a ser utilizada em um trabalho acadêmico deve ser técnico-científica e, portanto, formal (e não informal, como se o trabalho estivesse sendo explicado a um colega ou familiar). Desse modo, não devem ser usadas gírias. Além disso, não se deve escrever ``o trabalho feito por mim'', mas sim algo como ``o presente trabalho trata de/o trabalho realizado pelo(a) autor(a)/etc''.
	\item Utilize corretor ortográfico, verifique a gramática, e revise com muito cuidado e atenção todo o texto: pontuação, uso da vírgula, concordância, coesão e clareza textual, encadeamento entre frases, sentenças e parágrafos, etc.
	\item Este documento deve corresponder a uma monografia acadêmica, em que se deve fundamentar e justificar as ideias, afirmações, métodos e deduções apresentadas com base em técnicas, ferramentas, metodologias, equações, normas, literatura especializada (livros e artigos), etc, e não a um relatório descritivo voltado a um departamento/time de uma empresa. 
	\item As referências bibliográficas não podem conter apenas sites, blogs e manuais técnicos: devem também conter livros, artigos, dissertações, teses, etc. Além disso, há uma diferença entre citação \emph{direta} (comando \verb!\citeas!) e citação \emph{indireta} (comando \verb!\cite!).
\end{itemize}

\section*{Estrutura do documento}

Ao final da Introdução (que é o Capítulo 1), costuma-se apresentar como o documento está organizado, descrevendo brevemente o que é tratado nos demais capítulos do Sumário, começando pelo Capítulo 2. A estrutura de capítulos mostrada no Sumário deste documento é apenas uma sugestão geral e não precisa ser seguida à risca: dependendo da área e do foco do trabalho, tanto a estrutura detalhada (Capítulos, Seções e Subseções) quanto o número de capítulos pode variar. O(A) estudante deve consultar seu(sua) Orientador(a) Acadêmico(a) para definir a estrutura detalhada do documento.

\noindent\textbf{Exemplo}:

\emph{O presente documento está organizado da seguinte maneira. O Capítulo~2 apresenta a fundamentação teórica sobre os principais conceitos e técnicas necessárias para o entendimento do problema abordado e da solução proposta. O problema tratado neste PFC é descrito em detalhes no Capítulo~3, juntamente com os requisitos técnicos a serem atendidos. O Capítulo~4 aborda a solução proposta e a metodologia envolvida. O desenvolvimento realizado e a análise dos resultados obtidos são mostrados no Capítulo~5. Por fim, no Capítulo~6, são apresentadas as conclusões deste trabalho e algumas sugestões de trabalhos futuros são elencadas.}

