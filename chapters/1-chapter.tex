% ----------------------------------------------------------
\chapter{Introdução}
% ----------------------------------------------------------

% \textbf{Instruções do padrão genérico de TCCs da BU:} 

% As orientações aqui apresentadas são baseadas em um conjunto de normas elaboradas pela \gls{ABNT}. Além das normas técnicas, a Biblioteca também elaborou uma série de tutoriais, guias, \textit{templates} os quais estão disponíveis em seu site, no endereço \url{http://portal.bu.ufsc.br/normalizacao/}.

% Paralelamente ao uso deste \textit{template} recomenda-se que seja utilizado o \textbf{Tutorial de Trabalhos Acadêmicos} (disponível neste link \url{https://repositorio.ufsc.br/handle/123456789/180829}).

% Este \textit{template} está configurado apenas para a impressão utilizando o anverso das folhas, caso você queira imprimir usando a frente e o verso, acrescente a opção \textit{openright} e mude de \textit{oneside} para \textit{twoside} nas configurações da classe \textit{abntex2} no início do arquivo principal \textit{main.tex} \cite{abntex2classe}.

% Conforme a \href{https://repositorio.ufsc.br/bitstream/handle/123456789/197121/RN46.2019.pdf?sequence=1&isAllowed=y}{Resolução NORMATIVA nº 46/2019/CPG} as dissertações e teses não serão mais entregues em formato impresso na Biblioteca Universitária. Consulte o Repositório Institucional da UFSC ou sua Secretaria de Pós Graduação sobre os procedimentos para a entrega. 

% \nocite{NBR6023:2002}
% \nocite{NBR6027:2012}
% \nocite{NBR6028:2003}
% \nocite{NBR10520:2002}

% \textbf{Instruções da Coordenação do PFC:} 

% A Introdução deve apresentar um panorama geral do PFC de modo resumido, sem entrar em detalhes técnicos, para que o leitor já entenda aqui na Introdução o começo, meio e fim do PFC. Assim, é muito importante deixar bem claro na Introdução os seguintes pontos (a Introdução pode ser vista como um resumo geral do PFC, e a Seção Resumo como um resumo desse resumo): 

% \begin{itemize}
%      \item O contexto e a motivação do PFC, com uma breve descrição da empresa/instituto de pesquisa (histórico, clientes, produtos, serviços, projetos, etc) em que o PFC foi realizado e do projeto global da empresa em que o PFC está inserido (se for o caso)
%      \item Breve descrição do problema tratado no PFC;
%     \item A importância de tal problema para a empresa/clientes da empresa/instituto de pesquisa;
%     \item Objetivos: aqui são descritos os objetivos, que podem ser estratificados em objetivo geral e objetivos específicos. Outra opção é colocar os objetivos específicos na forma de passos de uma metodologia de trabalho, ou seja, os procedimentos e ferramentas adotadas em cada fase do projeto (um plano de trabalho).
%     \item Breve descrição da solução proposta;
%     \item Breve descrição da implementação/desenvolvimento realizado;
%     \item Breve descrição da metodologia e das principais técnicas e ferramentas utilizadas (do ponto de vista técnico). \textbf{Importante}: como \emph{Scrum} é uma metodologia geral de execução de projetos, ele não se enquadra aqui, pois não se trata de uma metodologia técnica específica para o desenvolvimento do PFC;
%     \item Breve descrição dos principais resultados obtidos e da importância/impactos de tais resultados para a empresa/clientes da empresa/instituto de pesquisa;
%     \item Deixar bem claro o que foi de fato foi realizado pelo(a) autor(a), diferenciando do que foi aproveitado de trabalhos anteriores/outras times da empresa. \textbf{Importante}: esta preocupação em diferenciar o trabalho realizado pelo(a) autor(a) daquele de possíveis colegas de um time deve permear todo o documento.
% \end{itemize}

% Ressaltamos a seguir alguns aspectos cruciais para a escrita do PFC:
% \begin{itemize}
% 	\item Apesar de o tamanho da monografia não ter uma correlação direta com a nota, bons trabalhos costumam ter de 60 a 70 páginas efetivamente escritas (desde a Introdução até a Conclusão, excluindo-se Capa, Resumo, Sumário, Referências Bibliográficas, etc).  Por outro lado, acima de 100 páginas a monografia pode se tornar ``massante'', discorrendo além do necessário para o entendimento do trabalho e, consequentemente, perdendo o foco do leitor.
% 	\item A linguagem a ser utilizada em um trabalho acadêmico deve ser técnico-científica e, portanto, formal (e não informal, como se o trabalho estivesse sendo explicado a um colega ou familiar). Desse modo, não devem ser usadas gírias. Além disso, não se deve escrever ``o trabalho feito por mim'', mas sim algo como ``o presente trabalho trata de/o trabalho realizado pelo(a) autor(a)/etc''.
% 	\item Utilize corretor ortográfico, verifique a gramática, e revise com muito cuidado e atenção todo o texto: pontuação, uso da vírgula, concordância, coesão e clareza textual, encadeamento entre frases, sentenças e parágrafos, etc.
% 	\item Este documento deve corresponder a uma monografia acadêmica, em que se deve fundamentar e justificar as ideias, afirmações, métodos e deduções apresentadas com base em técnicas, ferramentas, metodologias, equações, normas, literatura especializada (livros e artigos), etc, e não a um relatório descritivo voltado a um departamento/time de uma empresa. 
% 	\item As referências bibliográficas não podem conter apenas sites, blogs e manuais técnicos: devem também conter livros, artigos, dissertações, teses, etc. Além disso, há uma diferença entre citação \emph{direta} (comando \verb!\citeas!) e citação \emph{indireta} (comando \verb!\cite!).
% \end{itemize}

% \section*{Estrutura do documento}

% Ao final da Introdução (que é o Capítulo 1), costuma-se apresentar como o documento está organizado, descrevendo brevemente o que é tratado nos demais capítulos do Sumário, começando pelo Capítulo 2. A estrutura de capítulos mostrada no Sumário deste documento é apenas uma sugestão geral e não precisa ser seguida à risca: dependendo da área e do foco do trabalho, tanto a estrutura detalhada (Capítulos, Seções e Subseções) quanto o número de capítulos pode variar. O(A) estudante deve consultar seu(sua) Orientador(a) Acadêmico(a) para definir a estrutura detalhada do documento.

% \noindent\textbf{Exemplo}:

% \emph{O presente documento está organizado da seguinte maneira. O Capítulo~2 apresenta a fundamentação teórica sobre os principais conceitos e técnicas necessárias para o entendimento do problema abordado e da solução proposta. O problema tratado neste PFC é descrito em detalhes no Capítulo~3, juntamente com os requisitos técnicos a serem atendidos. O Capítulo~4 aborda a solução proposta e a metodologia envolvida. O desenvolvimento realizado e a análise dos resultados obtidos são mostrados no Capítulo~5. Por fim, no Capítulo~6, são apresentadas as conclusões deste trabalho e algumas sugestões de trabalhos futuros são elencadas.}

A situação atual das empresas que administram quadras esportivas revela uma crescente demanda por sistemas de agendamento e controle de reservas mais eficientes e automatizados. Com o aumento da competitividade no setor, essas empresas buscam soluções que não apenas organizem os horários e disponibilidades de seus espaços, mas que também integrem funcionalidades de automação de dispositivos, como iluminação, climatização e sistemas de acesso. A automação integrada aos agendamentos é crucial para reduzir desperdícios de energia e otimizar os custos operacionais, permitindo que os dispositivos sejam acionados apenas nos horários em que as quadras estão efetivamente em uso. Além disso, essas práticas têm um impacto positivo no meio ambiente, reduzindo a pegada de carbono das operações e promovendo um modelo de negócio mais sustentável.

\section{A Empresa Fischertec}

A Fischertec Tecnologia \cite{fischertec} é uma empresa dedicada a transformar ideias em soluções digitais de alta qualidade. Atua no desenvolvimento de sites, aplicativos, e-commerce e diversos produtos digitais, sempre focando em inovação e excelência.

A empresa conta com uma equipe de especialistas em design e desenvolvimento. Os designers são experientes em UX/UI e dedicados a criar interfaces que proporcionem a melhor experiência ao usuário. Os desenvolvedores fullstack da equipe da Fischertec têm expertise em diversas tecnologias e frameworks, garantindo a entrega de soluções robustas e eficazes.

A Fischertec, buscando criar um produto próprio com o objetivo de reduzir sua dependência de projetos externos, optou por desenvolver um sistema de gerenciamento de reservas de quadras esportivas. A decisão de desenvolver esse sistema foi motivada pela necessidade de atender demandas específicas das empresas esportivas, como personalização de horários, preços e integrações com dispositivos IoT. A criação de um sistema proprietário também reflete a estratégia da Fischertec de se posicionar como referência no setor, oferecendo uma solução robusta, eficiente e alinhada às melhores práticas de automação e gestão esportiva.

\section{O Problema e os Objetivos}

O problema central abordado é a dificuldade em gerenciar reservas de forma integrada e eficiente, definindo horários de funcionamento, preços de locação, tipos de quadras disponíveis entre outras opções. Essa dificuldade impacta diretamente a experiência dos clientes, bem como a eficiência administrativa da empresa.

O objetivo geral do projeto foi desenvolver um \gls{MVP} de uma plataforma que permita gerenciar reservas de quadras esportivas de forma centralizada, disponível na web e personalizável para diferentes empresas. Foram estabelecidos alguns objetivos específicos:

\begin{itemize}
    \item A criação de um sistema com autenticação de usuários;
    \item Controle de permissões de usuário baseado em papéis e configurável para diferentes \textit{tenants};
    % \item Definição e gerenciamento de quadras com possibilidade de personalização, como tipos de quadras, preços e horários de funcionamento;
    \item Criação, edição, busca e exclusão de quadras e armazenamento de informações como tipos de quadra, preço e horários de funcionamento;
    \item Configuração do \textit{tenant} da empresa, com nome e horários de funcionamento;
    \item Agendamento de reservas com controle de disponibilidade;
    \item Integração com dispositivos IoT da marca \cite{Sonoff} para automação de recursos presentes nas quadras de acordo com as reservas.
\end{itemize}

A Fischertec busca fornecer uma solução que otimize os processos operacionais e ofereça uma experiência mais eficiente e moderna para seus clientes empresariais, com foco nos esportistas que utilizam quadras esportivas e participam de aulas coletivas. Portanto, foi proposta uma solução que envolveu o desenvolvimento de uma aplicação web composta por um servidor com API REST e um sistema \textit{frontend}, ambos projetados para oferecer flexibilidade e desempenho. O sistema foi implementado utilizando uma arquitetura modular baseada em camadas, promovendo a separação de responsabilidades e facilitando futuras manutenções e expansões. Foram utilizadas tecnologias modernas, como Nest.js para a API, PostgreSQL para o banco de dados e Docker para conteinerização. A interface web será futuramente desenvolvida em Next.js/React. Além disso, houve integração com a plataforma IoT eWeLink da Shenzhen Coolkit Technology CO., LTD, possibilitando o controle automatizado de dispositivos associados às quadras.

A metodologia aplicada ao desenvolvimento seguiu práticas ágeis, garantindo entregas iterativas e validações frequentes. Ferramentas como Postman foram utilizadas para testes das APIs, enquanto o TypeORM auxiliou na persistência dos dados. O desenvolvimento foi estruturado de modo a criar um sistema totalmente funcional e alinhado com os objetivos propostos. Devido à maior complexidade, optou-se por focar no desenvolvimento da aplicação do servidor, que é o escopo deste PFC.

Os principais resultados obtidos incluem a entrega de uma aplicação \textit{backend} que permite o gerenciamento completo de locações, com controle de usuários e papéis para diferentes \textit{tentants} (locatários), gerenciamento de horários e automação de dispositivos IoT no início e fim das reservas. O sistema se mostrou eficiente e de fácil usabilidade, atendendo às expectativas da Fischertec ao simular um cenário real de operação. 

Todo o desenvolvimento da aplicação do servidor foi realizado pelo autor, sem aproveitamento direto de trabalhos prévios ou de outros times, o que reforça o caráter original e inovador do projeto. A aplicação cumpre o objetivo de resolver os problemas de gerenciamento enfrentados pela Fischertec, destacando-se como uma solução escalável e tecnológica para o setor.

\section{Estrutura do documento}

O presente documento está organizado da seguinte maneira. O \autoref{cap:fundamentacao_teorica} apresenta a fundamentação teórica sobre os principais conceitos e técnicas necessárias para o entendimento do problema abordado e da solução proposta. O problema tratado neste PFC é descrito em detalhes no \autoref{cap:descricao_problema_e_requisitos}, juntamente com os requisitos técnicos a serem atendidos. No \autoref{cap:solucao_proposta} é explicada a solução proposta e a metodologia utilizada. O \autoref{cap:tecnologias} mostra as ferramentas e tecnologias utilizadas. O \autoref{cap:desenvolvimento_e_analise_resultados} aborda o desenvolvimento da solução proposta e a análise dos resultados obtidos. Por fim, no \autoref{cap:conclusao}, são apresentadas as conclusões deste trabalho e algumas sugestões de trabalhos futuros são elencadas.