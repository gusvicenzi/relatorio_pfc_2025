% ----------------------------------------------------------
% \chapter{Desenvolvimento e Análise dos Resultados Obtidos}
\chapter{Análise dos Resultados Obtidos}
% ----------------------------------------------------------

\textbf{Instruções da Coordenação do PFC:}

Neste capítulo, deve-se:
\begin{itemize}
	\item Descrever detalhadamente como foi feita a implementação/desenvolvimento da solução proposta;
	\item Deixar bem claro e justificar tecnicamente para o leitor como que o desenvolvimento realizado implementa de fato a solução proposta, explicando tecnicamente as decisões que foram tomadas para se chegar a tal implentação;
	\item Analisar os resultados obtidos com base em indicadores, gráficos, estatísticas, etc: 
	\begin{itemize}
		\item A implementação realizada solucionou de fato o problema tratado? 
		\item Obteve-se o resultado esperado? 
		\item Mostrou-se melhor que o método anterior?
		\item Vantagens e desvantagens; 
		\item Problemas encontrados;   
		\item Impacto dos resultados obtidos nos processos/projetos/produtos/serviços/clientes da empresa/instituto de pesquisa; 
		\item Impactos organizacionais, tecnológicos, financeiros, éticos, ecológicos, etc.
	\end{itemize}
\end{itemize}

Sugere-se colocar uma diagrama/fluxograma ilustrando como que a solução proposta foi implementada/desenvolvida, e depois explicar em detalhes cada parte/bloco do diagrama/fluxograma ao longo do texto. 

Ressaltamos que, em princípio, existe uma infinidade de maneiras diferentes de implementar a solução proposta. Desse modo, o diagrama/fluxograma da solução proposta apresentado no capítulo anterior é mais geral e abstrato que o diagrama/fluxograma da implementação: a implementação realizada no PFC é uma maneira específica de se chegar à solução proposta a partir das técnicas, ferramentas e métodos utilizados. 

Alguns pontos que serão explicados nesse capítulo:

\begin{itemize}
	% \item \textbf{Ferramentas}: As tecnologias utilizadas na implementação da solução proposta, como linguagens de programação, frameworks, bibliotecas, etc.
	% \item \textbf{Arquitetura da solução}: A arquitetura utilizada na implementação da solução proposta, como monolítica, microsserviços, cliente-servidor, etc.
	% \item \textbf{Modelagem do banco de dados}: O modelo utilizado para representar os dados da aplicação, incluindo as entidades, relacionamentos, atributos, etc.
	% \item \textbf{Backend}: A parte da solução que lida com a manipulação dos dados e a regra de negócio da aplicação.
	% \subitem \textbf{Registro e autenticação de usuários}: Como o backend implementa a funcionalidade de registro e autenticação de usuários, incluindo validações, segurança, etc.
	% \subitem \textbf{Autorização de usuários}: Como o backend implementa a funcionalidade de autorização de usuários, incluindo permissões, roles, etc.
	% \subitem \textbf{Rotas da aplicação}: Como o backend implementa as rotas da aplicação, incluindo endpoints de CRUD para tenanants, quadras, agendamentos de horários e credenciais de acesso a API relacionada à internet das coisas (IoT).
	% \subitem \textbf{Detalhamento do agendamentos de horários}: Como o backend implementa a funcionalidade de agendamentos de horários de quadras, incluindo informações sobre os locais, os tipos de atividades, as datas e as horas.
	% \subitem \textbf{Detalhamento da integração com IoT}: Como o backend implementa a integração com dispositivos IoT e comanda o acionamento e desligamento de acordo com os agendamentos feitos pelosus usuários.
	\item \textbf{Análise de resultados}: Análise dos resultados obtidos com o desenvolvimento da aplicação. Detalhando os principais aspesctos relevantes, como a eficiência do agendamento de horários, a segurança das informações e a integração com IoT.
\end{itemize}