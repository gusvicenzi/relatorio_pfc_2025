% ----------------------------------------------------------
\chapter{Conclusão}\label{cap:conclusao}
% ----------------------------------------------------------

% \textbf{Instruções da Coordenação do PFC:}

% A Conclusão deve apresentar:
% \begin{itemize}
% 	\item Uma \textbf{síntese} do problema tratado, da solução proposta e dos principais resultados obtidos;
% 	\item Uma discussão sobre o que foi atingido no PFC em relação aos objetivos inicialmente traçados e sobre as limitações encontradas;
% 	\item Uma discussão sobre possíveis aprimoramentos;
% 	\item Destacar os impactos do PFC para a empresa/clientes da empresa/instituto de pesquisa, além de possíveis impactos organizacionais, tecnológicos, financeiros, éticos, ecológicos, etc. 
% 	\item Sugestão de trabalhos futuros (como se poderia dar continuidade ao PFC; aplicar o desenvolvimento realizado no PFC a outros problemas/processos; etc)
% \end{itemize}

\section{Síntese do Problema e Solução Proposta}


Com a crescente demanda por alugueis de quadras esportivas e a difusão do uso de dispositivos \acrshort{IoT} para controle de equipamentos elétricos e eletrônicos, surge a necessidade de uma solução robusta que permitisse a simplificação do processo de reserva de quadras e o controle dos equipamentos associados de maneira automatizada.
A Fischertec Tecnologia, buscando reduzir sua dependência de projetos de terceiros e analisando os desafios enfrentados pelas empresas na gestão de reservas de quadras esportivas e no controle automatizado de dispositivos \acrshort{IoT}, decidiu investir em uma solução própria para atender a essas necessidades do setor esportivo.

Este \acrshort{PFC} teve como objetivo solucionar os desafios enfrentados na gestão de reservas de quadras esportivas e no controle automatizado de dispositivos \acrshort{IoT}. Para isso, foi proposto o desenvolvimento de um sistema web \textit{multi-tenant} que permitisse o gerenciamento eficiente de reservas e a automação de dispositivos inteligêntes conectados via API eWeLink. A solução proposta incluiu um \textit{backend} robusto desenvolvido com tecnologias modernas como Nest.js, um \textit{frontend} intuitivo (que será desenvolvido em um futuro próximo) e uma integração eficiente e de fácil configuração com dispositivos \acrshort{IoT}, garantindo maior controle e otimização dos recursos.

\section{Avaliação dos Objetivos e Limitações Encontradas}

Os objetivos inicialmente traçados foram amplamente alcançados, com a criação de um sistema \textit{backend} funcional e eficiente para gestão de quadras e automação \acrshort{IoT}. A aplicação atendeu a todos os requisitos funcionais e não funcionais definidos nas Seções \ref{subsec:requisitos_funcionais} e \ref{subsec:requisitos_nao_funcionais}, respectivamente, proporcionando um ambiente seguro, escalável e acessível para os usuários.

Entretanto, algumas limitações foram identificadas ao longo do desenvolvimento. A dependência da API eWeLink para o controle dos dispositivos \acrshort{IoT} trouxe preocupações quanto à estabilidade da comunicação, custos de use e manutenção e o tempo de resposta das operações.

\section{Possíveis Aprimoramentos}

Para tornar a solução ainda mais eficiente, algumas melhorias podem ser exploradas no futuro:

\begin{itemize}
  \item Implementação de suporte a múltiplas APIs de dispositivos \acrshort{IoT}, reduzindo a dependência da eWeLink. Uma opção seria a integração com outras plataformas como Tuya, SmartLife, SmartThings e HomeAssistant.
  \item Otimização das consultas ao banco de dados para melhorar o desempenho do sistema, utilizando paginação para rotas com muitos registros.
  \item Uso de \acrfull{LLMs} para criação de relatórios detalhados sobre o uso das quadras e o consumo de energia.
  \item Desenvolvimento de um painel administrativo para monitoramento e controle em tempo real dos dispositivos.
\end{itemize}

\section{Sugestão de Trabalhos Futuros}

Para dar continuidade ao trabalho desenvolvido nesta \acrshort{PFC}, algumas direções podem ser exploradas:

\begin{itemize}
  \item Desenvolvimento da aplicação web \textit{frontend} para interação com o sistema já desenvolvido. Essa etapa acontecerá em uma etapa subsequente da \acrshort{PFC}.
  \item Desenvolvimento de um aplicativo móvel para facilitar o acesso ao sistema pelos clientes e administradores.
  \item Implementação de um módulo de notificações push para alertar os usuários sobre eventos importantes, como agendamentos próximos ou alterações no status dos dispositivos.
  \item Customização do sistema para atender aulas em grupo, com quantidade de vagas e associação de vários usuários ao agendamento.
  \item Desenvolvimento de um módulo de análise de dados para fornecer relatórios detalhados sobre a utilização das quadras.
  \item Integração com sistemas de pagamento online para facilitar a gestão financeira das reservas.
\end{itemize}

O desenvolvimento deste PFC demonstrou a viabilidade e os benefícios de uma solução integrada para a gestão de reservas de quadras esportivas e automação \acrshort{IoT}. Apesar dos desafios encontrados, os resultados obtidos comprovam a eficiência da abordagem adotada e abrem espaço para futuras melhorias e expansões. A criação de um sistema próprio representa um avanço para a Fischertec, permitindo maior autonomia através de faturamento recorrente além de levar ao mercado uma solução robusta que atenda às necessidades das empresas do setor esportivo.