% ----------------------------------------------------------
\chapter{Solução Proposta e Metodologia Utilizada}\label{cap:solucao_proposta}
% ----------------------------------------------------------

Este capítulo apresenta o processo de desenvolvimento do projeto de fim de curso relativo ao desenvolvimento de um sistema de gerenciamento de reservas e controle automatizado de dispositivos IoT para quadras esportivas. Na \autoref{sec:ferramentas_tecnologias} são apresentadas as ferramentas e tecnologias utilizadas no desenvolvimento do sistema. 


\textbf{Instruções da Coordenação do PFC:}

Neste capítulo, deve-se apresentar:
\begin{itemize}
	\item A solução proposta para o problema descrito no capítulo anterior;
	\item Explicar a metodologia utilizada na solução proposta;
	\item Deixar bem claro e justificar tecnicamente para o leitor como que a solução proposta resolve o problema abordado e atende aos requisitos técnicos, explicando tecnicamente as decisões que foram tomadas para se chegar a tal solução.
\end{itemize}

Sugere-se colocar uma diagrama/fluxograma/casos de uso/etc ilustrando a solução proposta, e depois explicar em detalhes cada parte/bloco do diagrama/fluxograma ao longo do texto. 

Ressaltamos que, em princípio, há uma infinidade de soluções possíveis para o problema abordado no PFC. Desse modo, é preciso explicar detalhadamente e justificar tecnicamente a solução proposta no PFC.

\section{Solução Proposta}

Para atender aos desafios identificados no \autoref{cap:descricao_problema_e_requisitos}, propõe-se o desenvolvimento de um sistema web baseado em \textit{cloud computing}, com uma abordagem de \textit{Software as a Service} (SaaS), voltado para o gerenciamento de reservas de quadras esportivas e integração com dispositivos IoT. Esse sistema será composto por um backend responsável por toda a lógica de negócios e armazenamento dos dados, e um frontend voltado para a interação com os clientes e a administração das quadras.

O sistema adotará uma abordagem multi-tenant, permitindo que múltiplas empresas utilizem a mesma plataforma de maneira independente. O administrador de cada empresa poderá configurar seu próprio ambiente, definindo nome, horários de funcionamento e quantidade de quadras disponíveis. Cada quadra terá horários de funcionamento ajustáveis de acordo com os dias da semana, garantindo flexibilidade na gestão.

Os clientes poderão acessar a plataforma para visualizar a disponibilidade das quadras em tempo real, selecionar um horário específico e realizar a reserva de forma simples e intuitiva. A confirmação das reservas será imediata por padrão, proporcionando uma experiência eficiente e conveniente tanto para os usuários quanto para os administradores. Será possível, também, configurar a necessidade de aprovação para as reservas, caso a empresa deseje revisar manualmente cada solicitação.

A automação do controle dos dispositivos IoT será feita por meio da integração com a plataforma eWeLink, amplamente utilizada no mercado e compatível com dispositivos da marca Sonoff. O administrador poderá conectar sua conta eWeLink ao sistema, permitindo a identificação automática dos dispositivos configurados. Esses dispositivos poderão ser atribuídos às quadras e configurados para executar comandos específicos no início e/ou fim de uma reserva, como ligar ou desligar iluminação e climatização de forma automatizada.

Com essa abordagem, o sistema proporcionará um gerenciamento eficiente e automatizado das quadras esportivas, reduzindo desperdícios, otimizando recursos e melhorando a experiência dos clientes. Além disso, a solução contribuirá para a redução de custos operacionais e impacto ambiental, tornando-se uma ferramenta indispensável para empresas que buscam modernização e eficiência em sua gestão.
