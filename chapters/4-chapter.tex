% ----------------------------------------------------------
% \chapter{Solução Proposta e Metodologia Utilizada}
\chapter{Desenvolvimento da Solução Proposta}\label{cap:solucao_proposta}
% ----------------------------------------------------------

Este capítulo apresenta o processo de desenvolvimento do projeto de fim de curso relativo ao desenvolvimento de um sistema de gerenciamento de reservas e controle automatizado de dispositivos IoT para quadras esportivas. Na \autoref{sec:ferramentas_tecnologias} são apresentadas as ferramentas e tecnologias utilizadas no desenvolvimento do sistema. 


\textbf{Instruções da Coordenação do PFC:}

Neste capítulo, deve-se apresentar:
\begin{itemize}
	\item A solução proposta para o problema descrito no capítulo anterior;
	\item Explicar a metodologia utilizada na solução proposta;
	\item Deixar bem claro e justificar tecnicamente para o leitor como que a solução proposta resolve o problema abordado e atende aos requisitos técnicos, explicando tecnicamente as decisões que foram tomadas para se chegar a tal solução.
\end{itemize}

Sugere-se colocar uma diagrama/fluxograma/casos de uso/etc ilustrando a solução proposta, e depois explicar em detalhes cada parte/bloco do diagrama/fluxograma ao longo do texto. 

Ressaltamos que, em princípio, há uma infinidade de soluções possíveis para o problema abordado no PFC. Desse modo, é preciso explicar detalhadamente e justificar tecnicamente a solução proposta no PFC.

Alguns pontos que serão explicados nesse capítulo:

\begin{itemize}
	\item \textbf{Ferramentas}: As tecnologias utilizadas na implementação da solução proposta, como linguagens de programação, frameworks, bibliotecas, etc.
	\item \textbf{Arquitetura da solução}: A arquitetura utilizada na implementação da solução proposta, como monolítica, microsserviços, cliente-servidor, etc.
	\item \textbf{Modelagem do banco de dados}: O modelo utilizado para representar os dados da aplicação, incluindo as entidades, relacionamentos, atributos, etc.
	\item \textbf{Backend}: A parte da solução que lida com a manipulação dos dados e a regra de negócio da aplicação.
	\subitem \textbf{Registro e autenticação de usuários}: Como o backend implementa a funcionalidade de registro e autenticação de usuários, incluindo validações, segurança, etc.
	\subitem \textbf{Autorização de usuários}: Como o backend implementa a funcionalidade de autorização de usuários, incluindo permissões, roles, etc.
	\subitem \textbf{Rotas da aplicação}: Como o backend implementa as rotas da aplicação, incluindo endpoints de CRUD para tenanants, quadras, agendamentos de horários e credenciais de acesso a API relacionada à internet das coisas (IoT).
	\subitem \textbf{Detalhamento do agendamentos de horários}: Como o backend implementa a funcionalidade de agendamentos de horários de quadras, incluindo informações sobre os locais, os tipos de atividades, as datas e as horas.
	\subitem \textbf{Detalhamento da integração com IoT}: Como o backend implementa a integração com dispositivos IoT e comanda o acionamento e desligamento de acordo com os agendamentos feitos pelosus usuários.
	\item \textbf{Análise de resultados}: Análise dos resultados obtidos com o desenvolvimento da aplicação. Detalhando os principais aspesctos relevantes, como a eficiência do agendamento de horários, a segurança das informações e a integração com IoT.
\end{itemize}

\section{Ferramentas e Tecnologias Utilizadas}\label{sec:ferramentas_tecnologias}

A seção de ferramentas e tecnologias utilizadas no desenvolvimento deste projeto apresenta um panorama detalhado dos recursos técnicos empregados para a construção do sistema de gerenciamento de reservas e controle automatizado de dispositivos IoT para quadras esportivas. 
% O projeto foi desenvolvido utilizando uma stack moderna e robusta, com o backend implementado em Nest.js, um framework eficiente e escalável para Node.js, e o frontend será desenvolvido em React, a fim de proporcionar uma experiência de usuário dinâmica e responsiva. Além disso, o banco de dados relacional PostgreSQL foi utilizado para garantir a integridade e a consistência dos dados, enquanto a integração com a plataforma eWeLink permitiu o controle automatizado de dispositivos IoT, agregando um nível adicional de automação e inovação ao sistema. Este conjunto de ferramentas foi escolhido com o objetivo de garantir um desenvolvimento ágil, eficiente e que atendesse às necessidades específicas do projeto.

\subsection{Git}

O Git é um sistema de controle de versão distribuído criado por Linus Torvalds em 2005, inicialmente para o desenvolvimento do kernel do Linux. Antes do Git, o projeto Linux utilizava o BitKeeper, mas devido a restrições de licenciamento, Torvalds decidiu desenvolver uma ferramenta própria. O objetivo principal era criar um sistema rápido, eficiente e que suportasse grandes projetos com facilidade.

As principais funções do Git incluem rastreamento de alterações no código-fonte, permitindo que múltiplos desenvolvedores trabalhem simultaneamente em um mesmo projeto sem conflitos. O Git possibilita a criação de ramificações (branches) para desenvolver novos recursos ou corrigir bugs de forma isolada, facilitando a integração dessas alterações ao projeto principal posteriormente. Além disso, como um sistema distribuído, cada desenvolvedor possui uma cópia completa do repositório, o que aumenta a segurança e a integridade dos dados. Atualmente, o Git é a ferramenta de controle de versão mais popular do mundo, amplamente utilizada por empresas de software, desenvolvedores independentes e projetos de código aberto.

Decidiu-se utilizar o Git devido a familiaridade da equipe com a ferramenta, já que é amplamente usada em outros projetos na empresa, além de sua robustez e confiabilidade.
 
O Git revolucionou a forma como o desenvolvimento de software é conduzido, proporcionando uma maneira eficiente e colaborativa de gerenciar projetos complexos. Sua flexibilidade, velocidade e robustez tornaram-no uma ferramenta essencial no arsenal de qualquer desenvolvedor, suportando desde pequenos projetos individuais até grandes sistemas empresariais.

\subsection{GitHub}

O GitHub é uma plataforma de hospedagem e repositório de código-fonte e arquivos que utiliza o Git para controle de versão. A plataforma permite que programadores ou qualquer usuário cadastrado contribuam em projetos, sejam eles privados ou de código aberto, de qualquer lugar do mundo. Amplamente adotada por desenvolvedores, o GitHub facilita a divulgação de trabalhos e a colaboração em projetos, além de oferecer recursos que simplificam a comunicação, como a identificação de problemas e a mesclagem de repositórios remotos por meio de issues e pull requests.

Atualmente, o GitHub é utilizado globalmente, contando com mais de 36 milhões de usuários ativos que contribuem em projetos comerciais ou pessoais. A plataforma hospeda mais de 100 milhões de projetos, incluindo alguns de renome mundial, como WordPress, GNU/Linux, Atom e Electron. Além disso, o GitHub oferece suporte a recursos de organização, amplamente utilizados por aqueles que desejam expandir seus projetos.

O Github é uma ferramenta usada por padrão na Fischertec. Embora outras plataformas como o Gitlab tenham funções semelhantes, optou-se por manter o padrão da empresa.

\subsection{PostgreSQL}

O SQL (Structured Query Language) foi desenvolvido nos anos 1970 pela IBM como uma linguagem padrão para gerenciar e manipular bancos de dados relacionais. PostgreSQL, um sistema de gerenciamento de banco de dados relacional, surgiu na Universidade da Califórnia em Berkeley, no final dos anos 1980, como parte do projeto POSTGRES (Post Ingres). Em 1996, o sistema foi renomeado para PostgreSQL, refletindo sua compatibilidade com SQL. Desde então, PostgreSQL evoluiu significativamente, tornando-se uma das bases de dados relacionais mais robustas e avançadas disponíveis.

PostgreSQL oferece uma ampla gama de funções, incluindo suporte a transações ACID (Atomicidade, Consistência, Isolamento, Durabilidade), integridade referencial, e extensões para manipulação de dados complexos como JSON e XML. Além disso, suporta procedimentos armazenados, gatilhos e uma variedade de tipos de índices para otimizar consultas. Atualmente, PostgreSQL é amplamente utilizado por grandes empresas, startups e projetos de código aberto devido à sua confiabilidade, flexibilidade e conformidade com os padrões SQL.

Existem outras soluções de banco de dados SQL como MySQL, que é conhecido por sua facilidade de uso e performance em aplicações web, e soluções NoSQL como MongoDB, que é ideal para armazenar grandes volumes de dados não estruturados e aplicações que exigem alta escalabilidade e flexibilidade. Optou-se por usar o PostgreSQL no desenvolvimento do projeto devido à sua capacidade de lidar com transações complexas, suporte a dados estruturados e não estruturados, e forte conformidade com os padrões SQL, o que garante integridade e consistência dos dados.

O PostgreSQL se destaca como uma solução de banco de dados robusta e versátil, adequada para uma ampla gama de aplicações. Sua evolução ao longo dos anos e a capacidade de suportar funcionalidades avançadas o tornam uma escolha excelente para projetos que exigem confiabilidade e flexibilidade. A decisão de utilizá-lo no projeto foi fundamentada em sua capacidade de atender às necessidades específicas de gerenciamento de dados de maneira eficiente e segura.

\subsection{Typescript}

O JavaScript, criado em 1995 por Brendan Eich da Netscape, rapidamente se tornou a linguagem padrão para desenvolvimento web, permitindo a criação de páginas dinâmicas e interativas. No entanto, à medida que aplicações web se tornavam mais complexas, as limitações de JavaScript em termos de tipagem estática e suporte para grandes projetos se tornaram evidentes. Para reduzir essas limitações, a Microsoft lançou o TypeScript em 2012, uma linguagem de programação de código aberto que é um superconjunto (superset) estrito de JavaScript. TypeScript adiciona tipagem estática e outros recursos avançados ao JavaScript, ajudando os desenvolvedores a escrever código mais seguro e escalável.

As principais funções do TypeScript incluem a adição de tipos estáticos, interfaces, classes e módulos, que não existem nativamente no JavaScript. Essas funcionalidades permitem a detecção de erros durante o desenvolvimento, antes do tempo de execução, melhorando a qualidade do código. Além disso, TypeScript se transpila (é convertido) para JavaScript, garantindo compatibilidade total com os navegadores e plataformas que suportam JavaScript. Atualmente, TypeScript é amplamente adotado em projetos de grande escala devido à sua capacidade de melhorar a produtividade e a manutenção do código. Empresas como Google, Microsoft e Airbnb utilizam TypeScript em seus projetos.

No desenvolvimento do projeto backend, optou-se por TypeScript em vez de JavaScript por várias razões. TypeScript proporciona uma melhor experiência de desenvolvimento, oferecendo autocompletar, navegação de código e verificação de tipos em tempo de compilação, o que ajuda a evitar muitos erros comuns em JavaScript. Isso é especialmente importante em projetos backend, onde a robustez e a previsibilidade do código são cruciais. Além disso, TypeScript facilita a colaboração entre desenvolvedores, permitindo que eles entendam e mantenham o código com mais facilidade.

O TypeScript se destaca como uma ferramenta poderosa que complementa e aprimora JavaScript, tornando o desenvolvimento de software mais eficiente e menos propenso a erros. Sua adoção no desenvolvimento da aplicação backend de gerenciamento de reservas e controle automatizado de dispositivos IoT permitiu criar um código mais robusto e sustentável.

\subsection{Node.js}

O Node.js é uma plataforma de desenvolvimento open-source baseada no motor JavaScript V8 do Google Chrome, criada em 2009 por Ryan Dahl. Seu objetivo inicial era possibilitar a execução de JavaScript no lado do servidor, permitindo o desenvolvimento de aplicações web com maior eficiência e escalabilidade. Desde seu lançamento, o Node.js rapidamente ganhou popularidade devido à sua arquitetura orientada a eventos e sua capacidade de lidar com operações de entrada e saída de forma não bloqueante, o que é particularmente útil para aplicações em tempo real.

Atualmente, o Node.js é amplamente utilizado para construir desde APIs e microservices até aplicações completas de grande escala. Sua capacidade de executar JavaScript tanto no cliente quanto no servidor facilita o desenvolvimento fullstack, enquanto a vasta biblioteca de pacotes disponíveis através do npm (Node Package Manager) oferece soluções para praticamente qualquer necessidade de desenvolvimento. O Node.js tem sido uma escolha popular em diversas indústrias devido à sua performance, escalabilidade e ao suporte contínuo de uma grande comunidade de desenvolvedores.

Uma das principais vantagens do Node.js em comparação com outras soluções é sua natureza assíncrona e orientada a eventos, que permite o gerenciamento eficiente de múltiplas conexões simultâneas sem sobrecarregar os recursos do servidor. Além disso, a utilização de JavaScript, uma linguagem amplamente conhecida e utilizada, reduz a curva de aprendizado para novos desenvolvedores e facilita a integração entre equipes de frontend e backend. A modularidade do Node.js e sua grande comunidade de apoio também contribuem para o desenvolvimento mais rápido e eficiente de aplicações.

Portanto o Node.js se destaca como uma solução versátil e eficiente para o desenvolvimento de aplicações modernas. Sua combinação de performance, escalabilidade e uma vasta gama de ferramentas e bibliotecas tornam-no uma escolha robusta para desenvolvedores que buscam criar aplicações rápidas e escaláveis, atendendo às demandas de um mercado cada vez mais dinâmico e competitivo.

\subsection{Nest.js}

O Nest.js é um framework de desenvolvimento backend criado em 2017 por Kamil Myśliwiec. Inspirado nos princípios de programação modular e fortemente influenciado pelo Angular, o Nest.js foi projetado para oferecer uma estrutura robusta e escalável para a construção de aplicações do lado do servidor. Desde o seu lançamento, o framework tem crescido em popularidade, especialmente entre desenvolvedores que buscam uma abordagem moderna para o desenvolvimento de aplicações backend em Node.js.

O Nest.js fornece uma arquitetura modular que facilita a criação e a manutenção de aplicações escaláveis e bem estruturadas. Ele suporta diversos paradigmas de programação, incluindo orientação a objetos, programação funcional e reativa. Com suporte nativo para TypeScript, o Nest.js oferece tipagem estática, o que melhora a confiabilidade e a legibilidade do código. Atualmente, é amplamente utilizado para construir APIs RESTful, microservices e aplicativos monolíticos, sendo uma escolha frequente em projetos que exigem alta escalabilidade e flexibilidade.

Entre as principais vantagens do Nest.js estão sua modularidade, que permite a criação de aplicações altamente organizadas e de fácil manutenção, e seu suporte integral a TypeScript, que traz maior segurança e produtividade no desenvolvimento. Em comparação com outros frameworks, como Express.js, o Nest.js oferece uma estrutura mais robusta e orientada a boas práticas, facilitando o desenvolvimento de aplicações complexas. No contexto do projeto de fim de curso, o Nest.js foi escolhido por sua capacidade de suportar a criação de uma aplicação multi-tenant complexa, com necessidades específicas de escalabilidade, organização e integração com outras tecnologias.

O Nest.js se destaca como uma ferramenta poderosa para o desenvolvimento de aplicações backend modernas, combinando a performance do Node.js com uma arquitetura flexível e escalável. Sua adoção neste projeto de fim de curso reflete a busca por soluções eficientes e de alta qualidade, alinhadas com as demandas atuais do mercado de tecnologia.

\subsection{Docker}

O Docker foi lançado em 2013 por Solomon Hykes, inicialmente como um projeto interno da empresa dotCloud. A plataforma foi criada com o objetivo de simplificar o processo de desenvolvimento, distribuição e execução de aplicações por meio da virtualização de contêineres. Desde seu lançamento, o Docker tem revolucionado a maneira como aplicações são desenvolvidas e implantadas, tornando-se uma das ferramentas mais populares no ecossistema de DevOps.

Atualmente, o Docker é amplamente utilizado para criar, distribuir e executar aplicações em contêineres, que são ambientes leves e portáteis que contêm tudo o que uma aplicação precisa para ser executada. Isso inclui código, bibliotecas e dependências, garantindo que a aplicação funcione de maneira consistente em qualquer ambiente. O Docker é uma escolha comum em projetos que exigem portabilidade, escalabilidade e uma integração contínua eficiente, sendo adotado por empresas de todos os tamanhos para modernizar seus fluxos de trabalho de desenvolvimento e implantação.

Entre as principais vantagens do Docker estão a portabilidade de aplicações, a eficiência no uso de recursos e a facilidade de integração com pipelines de CI/CD (integração contínua e entrega contínua). Comparado com máquinas virtuais tradicionais, os contêineres do Docker são mais leves e rápidos, o que resulta em tempos de inicialização menores e melhor utilização de recursos. No contexto deste projeto, o Docker foi escolhido para garantir que o ambiente de desenvolvimento e produção seja consistente, facilitando a implantação em servidores AWS e reduzindo problemas de compatibilidade.

O Docker se destaca como uma solução essencial para o desenvolvimento e a implantação de aplicações modernas, oferecendo uma maneira eficiente de gerenciar ambientes de software. Sua escolha neste projeto reflete a busca por uma solução que ofereça portabilidade, eficiência e flexibilidade, alinhando-se às melhores práticas do mercado e garantindo um processo de desenvolvimento mais ágil e confiável.

\subsection{Postman}

O Postman foi lançado em 2012 por Abhinav Asthana como uma ferramenta auxiliar para o desenvolvimento de APIs. Inicialmente criado como uma extensão para o Google Chrome, o Postman rapidamente evoluiu para uma aplicação completa e independente, tornando-se uma das ferramentas mais populares para o teste e desenvolvimento de APIs. Ao longo dos anos, a plataforma expandiu suas funcionalidades para atender às crescentes demandas de desenvolvedores e equipes de API.

O Postman é amplamente utilizado para testar, documentar e monitorar APIs, oferecendo uma interface amigável que facilita a realização de requisições HTTP, o gerenciamento de coleções de APIs e a automatização de testes. Além disso, ele permite a colaboração em equipe, possibilitando o compartilhamento de coleções e ambientes de teste. Atualmente, o Postman é uma ferramenta indispensável no fluxo de trabalho de desenvolvedores e engenheiros de qualidade, sendo utilizado por milhões de usuários ao redor do mundo.

As vantagens do Postman incluem sua interface intuitiva, que reduz a complexidade do teste de APIs, e suas capacidades avançadas de automação e documentação. Comparado a outras ferramentas, como cURL ou alternativas mais básicas, o Postman oferece uma experiência mais integrada e acessível, especialmente para equipes que precisam colaborar em projetos complexos. No desenvolvimento deste projeto, o Postman foi escolhido por sua facilidade de uso e suas capacidades de automação de testes, o que contribui para um processo de desenvolvimento mais eficiente e menos propenso a erros.

Portanto o Postman se consolida como uma ferramenta essencial para o desenvolvimento e manutenção de APIs, oferecendo funcionalidades abrangentes que facilitam o trabalho de desenvolvedores e equipes de qualidade. Sua utilização neste projeto destaca a importância de ferramentas que otimizam o fluxo de trabalho, garantindo maior eficiência e qualidade no desenvolvimento de sistemas modernos.